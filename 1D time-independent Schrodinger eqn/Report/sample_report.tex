%
\documentclass[12pt]{iopart}
\newcommand{\gguide}{{\it Preparing graphics for IOP journals}}
%Uncomment next line if AMS fonts required
\usepackage{iopams}  
\usepackage{graphicx}
\usepackage{subfigure}
%\usepackage{epsfig}

\begin{document}

\title[PHY 711: Sample report]{Chaotic behavior and the Lorenz equations}

\author{Michael D. Vera}
\address{The University of Southern Mississippi,
Department of Physics and Astronomy, Hattiesburg, Mississippi, 
39406}
\ead{michael.vera@usm.edu}

\bibliographystyle{softe1lwbosa}

\begin{abstract}

The Lorenz equations were developed as a simplified model of thermally driven 
convection in the atmosphere.  These equations are expressed as the first-order 
time derivatives of three variables.  Solutions to the Lorenz equations can 
exhibit highly sensitive, chaotic behavior.  For a particular set of control 
parameters, trajectories in the three variables, orbits in phase space and the 
differences resulting from slightly different sets of initial conditions are 
presented.

\end{abstract}

\section{Introduction}
\label{intro}

Lorenz examined the behavior of the following system 
of first-order differential equations, developed as a highly simplified 
model of thermally driven atmospheric convection,

\begin{eqnarray}
\label{eqlorenz}
\frac{dx}{dt}=\sigma (y-x) \nonumber \\
\frac{dy}{dt}=rx-y-xz \nonumber \\
\frac{dz}{dt}=xy-bz \nonumber \\
\end{eqnarray}

\noindent The parameters $\sigma$, $r$ and $b$, as well as initial 
conditions for $(x,y,z)$, must be provided in order to treat this system 
using initial value techniques.  For this project, the values of $(\sigma,r,b)$ 
are $(10,28,8/3)$.  These are the values originally explored by Lorenz.\cite{davies}

Due to the nonlinear terms in the second and third lines in Equation \ref{eqlorenz}, 
this system has the potential to exhibit chaotic behavior.  Chaotic behavior involves 
the nonperiodic behavior of deterministic systems and a pronounced sensitivity to initial 
conditions that is sometimes referred to as the ``butterfly effect".\cite{davies}  This 
sensitivity can be demonstrated in the Lorenz equations by examining the differences 
between solutions generated by slightly different sets of initial conditions.

One property possessed by a variety of chaotic systems is the evolution toward an 
``attractor".  An ``attractor" is a geometrical object toward which systems beginning 
within some range of parameters are drawn.  After an initial approach, then, longer-time 
depictions of the behavior of such systems can serve to map the attractor.  The attractor 
for the Lorenz equations has become a well known example of chaotic dynamics.

\section{Methods}
\label{methods}

Given a set of initial conditions ($x_0$, $y_0$ and $z_0$), solutions 
to the Lorenz equations can be generated by incrementing the variables forward over 
a time step, $\Delta t$, by an amount that depends on the derivatives given by 
Equation \ref{eqlorenz}.  
These solutions to the Lorenz equations will be developed using fourth-order Runge-Kutta 
techniques.  In order to advance the values of the variables through time, each of the 
three is 
updated with the following equation.

\begin{equation}
\label{eqrk4}
y_{i+1}=y_i+\frac{1}{6}(k_1+2k_2+2k_3+k_4)+O(\Delta t^5)
\end{equation}

\noindent At the beginning of the time step, the variables have the value 
$y_i$.  The updated variables at the end of the time step are changed by a weighted 
average of various estimates of their change in value.  The error in executing this 
step can be characterized as fifth order in the step size; therefore, this method 
is accurate to fourth order.  The first derivative equations 
are used to construct the estimated changes in the following way.\cite{numrec,koonin}

\begin{eqnarray}
\label{eqrk4lst}
k_1=\Delta t [f(t_i,y_i)] \nonumber \\
k_2=\Delta t [f(t_i+\frac{\Delta t}{2},y_i+\frac{k_1}{2})] \nonumber \\
k_3=\Delta t [f(t_i+\frac{\Delta t}{2},y_i+\frac{k_2}{2})] \nonumber \\
k_4=\Delta t [f(t_i+\Delta t,y_i+k_3)] \nonumber \\
\end{eqnarray}

\noindent The function, $f(t,y)$, is the first derivative for the variable $y$ according 
to Equation \ref{eqlorenz}.  Note that, in this case, there is no dependence of the 
first derivative on the independent variable $t$.

In order to investigate the behavior of solutions to the Lorenz equations, solutions 
are developed for $0<t<50$ in time steps of $0.001$.  The primary initial 
conditions used are $(x_0,y_0,z_0)=(0,1,0)$; the resulting solutions can be compared 
to the behavior when $y_0$ is changed by $10^{-8}$.

%----------------------------------------------------------------------------
\begin{figure}
\includegraphics[width=6in]{figures/lor4_4a_xz.eps}
\caption{\label{figlorenz}
The red curve depicts $z$ vs. $x$ for $(\sigma,r,b)=(10,28,\frac{8}{3})$ for times 
from $0$ to $50$ with $(x_0,y_0,z_0)=(0,1,0)$.  The location at $t=50$ is marked 
with a red dot.  A computation that differed only by using $y_0=1+10^{-8}$ results 
in the $t=50$ location marked with a blue dot. 
}
\end{figure}
%----------------------------------------------------------------------------

\section{Results}
\label{results}

A sample of the behavior of this system is depicted 
in Figure \ref{figlorenz} for $(x_0,y_0,z_0)=(0,1,0)$ and $(\sigma,r,b)$ 
are $(10,28,8/3)$.
With these initial conditions, the system starts well outside the region of the 
joined shapes but converges toward them.  Once in the vicinity, the system orbits 
one of the oval structures some number of times then switches to the other one.  This 
elaborate geometrical structure also constitutes an ``attractor".
The attractor can be depicted more clearly by discarding the first $20000$ time steps, 
so that only $20<t<50$ are shown, as in Figure \ref{figatt}.  Then, the approach to 
the attractor is not included.

%----------------------------------------------------------------------------
\begin{figure}
\includegraphics[width=6in]{figures/attractor.eps}
\caption{\label{figatt}
The red curve depicts $z$ vs. $x$ for $(\sigma,r,b)=(10,28,\frac{8}{3})$ for times 
from $20$ to $50$ with $(x_0,y_0,z_0)=(0,1,0)$.  For these times, the approach to the 
attractor has already been completed and subsequent evolution serves to illustrate 
its structure. 
}
\end{figure}
%----------------------------------------------------------------------------

Though the evolution of the dynamical system is drawn toward this ``attractor" structure, 
there is also a sensitive dependence on initial conditions.  Even for the extremely 
small difference in initial conditions $\Delta y_0=10^{-8}$, the final locations of the 
system (indicated by points in Figure \ref{figlorenz}) are dramatically different.  
Another way to depict this sensitivity is to compare the coordinates 
as a function of time.    
The value of $y$ as a function 
of time for each of the slightly different systems is shown in Figure \ref{figloryt}.

%----------------------------------------------------------------------------
\begin{figure}
\includegraphics[width=6in]{figures/lor4_4a_yt.eps}
\caption{\label{figloryt}
The red curve depicts $y$ vs. $t$ with $(x_0,y_0,z_0)=(0,1,0)$.  
The blue curve has very slightly different initial conditions 
with $\Delta y_0=1+10^{-8}$. Though the solutions are indistinguishable at 
early times, they ultimately diverge.
}
\end{figure}
%-----------------------------------------------------------------------------

One way to depict the evolution of dynamical systems is to display the values 
of the relevant variables, as is done in Figures \ref{figatt} and \ref{figloryt}.  
Phase space provides another way of representing a system's evolution.  For a 
system with $N$ variables, the entire 
space is $2N$ dimensional consisting of the value and the derivative of each 
quantity.  The Lorenz system, with three variables, therefore has a six-dimensional 
phase space but a two-dimensional representation can be made for each 
coordinate by plotting the coordinate's value and its rate of change.  A plot of 
$(y,\frac{dy}{dt})$ are shown in Figure \ref{figphase}.  As before, the beginning of 
the evolution is discarded and only $20<t<50$ is displayed.

%----------------------------------------------------------------------------
\begin{figure}
\includegraphics[width=6in]{figures/lorph.eps}
\caption{\label{figphase}
A two-dimensional projection of the six-dimensional phase space is presented.  
The values of $(y,\frac{dy}{dt})$ are shown for $20<t<50$ with $(x_0,y_0,z_0)=(0,1,0)$.  
The complex character of the motion is evident.
}
\end{figure}
%-----------------------------------------------------------------------------

\section{Conclusion}

The Lorenz equations have been examined by generating solutions for control 
parameters $(\sigma,r,b)=(10,28,8/3)$ using fourth-order Runge-Kutta integration 
techniques.  Two slightly different sets of initial conditions were considered, 
$(x_0,y_0,z_0)=(0,1,0)$ and $(x_0,y_0,z_0)=(0,1+10^{-8},0)$.  The chaotic sensitivity of 
the system of equations was demonstrated as these two sets of initial parameters led 
to divergent solutions for $t$ greater than about $35$.

A particular projection of the attractor governing the evolution of the system was 
depicted in Figure \ref{figatt}.  After excluding an initial time that represents the 
approach to the attractor, the system's subsequent behavior served to map the attractor's 
structure.  The path of a solution in phase space, the six-dimensional space consisting 
of the variables and their derivatives, was considered and a two-dimensional projection 
was presented in Figure \ref{figphase}.

Other areas that could be investigated include the numerical properties of the integration 
algorithm.  The importance of step size and the implementation of error monitoring might 
serve to protect against the possibility of purely artificial effects.  The process of 
constructing Poincar\'{e} sections could also be investigated.  This involves plotting a 
phase space point whenever the system intersects some value of a variable.  One potential 
choice would be to plot the phase-space location of the system whenever $x=0$ and 
$\frac{dx}{dt}>0$.

\pagebreak

\bibliography{sample_report}

\end{document}

